
% Don't know how to obtain prefixsegmented numberstring given refstring
% Once known, can alter inrange.lua to handle this case (hook
% points marked with TODO in that file)
% Until then, no need for setupcaption with prefixsegments
% \setupcaption[figure][
    %prefix=yes, 
    %prefixsegments=section:subsection, 
    %way=bysubsection]

\usemodule[inrange]
\setupinteraction[state=start]

\starttext

A complex set of ranges, with two unknown references, a prefix, and a
suffix:
See \inrange{Figures}{...}[
    fig:a, 
    fig:c, 
    fig:d, 
    fig:e, 
    fig:g, 
    fig:x, 
    fig:h]

Two figures, not a range. See \inrange{figures}[fig:a, fig:c].

Two figures, adjacent.
See \inrange{figures}[fig:a, fig:b].

One figure, no range at all.
See \inrange{figure}[fig:a].

\hairline

\definereferenceformat[inr][left=«, right=», command=\inrange]

\inr{Figures }[fig:g, fig:b].


\section{asdf}
\subsection{asdf-a}
    \placefigure[here][fig:a]{aaa}{
    \startformula
        1 + 1 = 2
    \stopformula
    }

    \placefigure[here][fig:b]{bbb}{
    \startformula
        1 + 1 = 2
    \stopformula
    }
    \placefigure[here][fig:c]{ccc}{
    \startformula
        1 + 1 = 2
    \stopformula
    }
    \placefigure[here][fig:d]{ddd}{
    \startformula
        1 + 1 = 2
    \stopformula
    }
    \placefigure[here][fig:e]{eee}{
    \startformula
        1 + 1 = 2
    \stopformula
    }
    \placefigure[here][fig:f]{fff}{
    \startformula
        1 + 1 = 2
    \stopformula
    }
    \placefigure[here][fig:g]{ggg}{
    \startformula
        1 + 1 = 2
    \stopformula
    }

%\startluacode
    %a = commands.savedlistprefixednumber(nil,6)}
%\stopluacode


\stoptext
